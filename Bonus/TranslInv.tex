
% !TeX spellcheck = en_US
\documentclass{article}
\usepackage[english]{babel}
\usepackage[utf8]{inputenc}
\usepackage{fancyhdr}
\usepackage{xcolor}
\usepackage{lmodern}
\usepackage{listings}
\usepackage{amsmath}
\usepackage{amssymb}
\usepackage{graphicx}
\usepackage{physics}
\lstset{language=[90]Fortran,
	basicstyle=\ttfamily,
	keywordstyle=\color{blue},
	commentstyle=\color{green},
	morecomment=[l]{!\ }% Comment only with space after !
}

\usepackage{color}
\definecolor{deepblue}{rgb}{0,0,0.5}
\definecolor{deepred}{rgb}{0.6,0,0}
\definecolor{deepgreen}{rgb}{0,0.5,0}

% Default fixed font does not support bold face
\DeclareFixedFont{\ttb}{T1}{txtt}{bx}{n}{10} % for bold
\DeclareFixedFont{\ttm}{T1}{txtt}{m}{n}{10}  % for normal

% Python style for highlighting
\lstset{
	language=Python,
	basicstyle=\ttm,
	otherkeywords={self},             % Add keywords here
	keywordstyle=\ttb\color{deepblue},
	emph={__init__},          % Custom highlighting
	emphstyle=\ttb\color{deepred},    % Custom highlighting style
	stringstyle=\color{deepgreen},
	frame=tb,                         % Any extra options here
	showstringspaces=false            % 
}





\pagestyle{fancy}
\fancyhf{}
\lhead{Vincenzo Maria Schimmenti - 1204565}
\rhead{\today}
\rfoot{Page \thepage}
\lfoot{Exercise 9bis}
\title{%
	Information Theory and Computation \\
	Exercise  9bis}
\author{Vincenzo Maria Schimmenti - 1204565}
\begin{document}
\maketitle
 
\section*{Translational invariant states}
Take the translational operator $T=\sum_{i=1}^N \sigma_x^i \sigma_x^{i+1}$ (using periodic boundary conditions). Suppose we want to diagonalize this operator: we need to build $2^N$ states that are traslational invariant. Suppose we find an ensemble of $M$ states $\{ t_i \}$ for which
\begin{equation}
	T t_i = t_{i+1}
\end{equation}
then  any Fourier like combination is traslational invariant:
\begin{equation}
	T v_p = T \sum_{q=1}^M e^{2\pi i \frac{pq}{M}} t_q = \sum_{q=1}^M e^{2\pi i \frac{pq}{M}} t_{q+1} = e^{-2\pi i \frac{p}{M}} v_p
\end{equation}
We have to build $2^N$ of these states and we have to do it in a way that exploits the above property. Let's start from the trivial state which is $\ket{0 \dots 0}$. Now we proceed by adding one $1$ at the previous state i.e. we use as $\{ t_i \}$ the followings states:
\begin{equation}
	\ket{100 \dots 00}, \ket{010 \dots 00}, \ket{001 \dots 00}, \dots, \ket{000 \dots 01}
\end{equation}
There exist $N$ states like this and using the Fourier series we can extract $N$ basis vector. In this way we are block-diagonalizing the Hamiltonian. Let's move to the next kind of state by adding another $1$:
\begin{flalign}
	& \ket{1100 \dots 00}, \ket{0110 \dots 00}, \ket{0011 \dots 00}, \dots, \ket{000 \dots 11} \\
	& \ket{10100 \dots 000}, \ket{01010 \dots 000}, \ket{00101 \dots 000}, \dots, \ket{000 \dots 101}, \ket{010 \dots 001} \\
	& \ket{100100 \dots 000}, \ket{010010 \dots 000}, \ket{001001 \dots 000}, \dots, \ket{100000 \dots 100}, \ket{010000 \dots 010}, \ket{001000 \dots 001}
\end{flalign}
\end{document}